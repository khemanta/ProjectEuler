Template for the AMS artcle style.


%%
%% This is LaTeX2e input.
%%

%% The following tells LaTeX that we are using the 
%% style file amsart.cls (That is the AMS article style
%%
\documentclass{amsart}

%% This has a default type size 10pt.  Other options are 11pt and 12pt
%% This are set by replacing the command above by
%% \documentclass[11pt]{amsart}
%%
%% or
%%
%% \documentclass[12pt]{amsart}
%%

%%
%% Some mathematical symbols are not included in the basic LaTeX
%% package.  Uncommenting the following makes more commands
%% available. 
%%

%\usepackage{amssymb}

%%
%% The following is commands are used for importing various types of
%% grapics.
%% 

%\usepackage{epsfig}  		% For postscript
%\usepackage{epic,eepic}       % For epic and eepic output from xfig

%%
%% The following is very useful in keeping track of labels while
%% writing.  The variant   \usepackage[notcite]{showkeys}
%% does not show the labels on the \cite commands.
%% 

%\usepackageshowkeys}


%%%%
%%%% The next few commands set up the theorem type environments.
%%%% Here they are set up to be numbered section.number, but this can
%%%% be changed.
%%%%

\newtheorem{thm}{Theorem}[section]
\newtheorem{prop}[thm]{Proposition}
\newtheorem{lem}[thm]{Lemma}
\newtheorem{cor}[thm]{Corollary}


%%
%% If some other type is need, say conjectures, then it is constructed
%% by editing and uncommenting the following.
%%

%\newtheorem{conj}[thm]{Conjecture} 


%%% 
%%% The following gives definition type environments (which only differ
%%% from theorem type invironmants in the choices of fonts).  The
%%% numbering is still tied to the theorem counter.
%%% 

\theoremstyle{definition}
\newtheorem{definition}[thm]{Definition}
\newtheorem{example}[thm]{Example}

%%
%% Again more of these can be added by uncommenting and editing the
%% following. 
%%

%\newtheorem{note}[thm]{Note}


%%% 
%%% The following gives remark type environments (which only differ
%%% from theorem type invironmants in the choices of fonts).  The
%%% numbering is still tied to the theorem counter.
%%% 


\theoremstyle{remark}

\newtheorem{remark}[thm]{Remark}


%%%
%%% The following, if uncommented, numbers equations within sections.
%%% 

\numberwithin{equation}{section}


%%%
%%% The following show how to make definition (also called macros or
%%% abbreviations).  For example to use get a bold face R for use to
%%% name the real numbers the command is \mathbf{R}.  To save typing we
%%% can abbreviate as

\newcommand{\R}{\mathbf{R}}  % The real numbers.

%%
%% The comment after the defintion is not required, but if you are
%% working with someone they will likely thank you for explaining your
%% definition.  
%%
%% Now add you own definitions:
%%

%%%
%%% Mathematical operators (things like sin and cos which are used as
%%% functions and have slightly different spacing when typeset than
%%% variables are defined as follows:
%%%

\DeclareMathOperator{\dist}{dist} % The distance.



%%
%% This is the end of the preamble.
%% 


\begin{document}

%%
%% The title of the paper goes here.  Edit to your title.
%%

\title{Mathematical Theorems}

%%
%% Now edit the following to give your name and address:
%% 

\author{Ralph Howard}
\address{Department of Mathematics, University of South Carolina, 
Columbia, SC 29208}
\email{howard@math.sc.edu}
\urladdr{www.math.sc.edu/$\sim$howard} % Delete if not wanted.

%%
%% If there is another author uncomment and edit the following.
%%

%\author{Second Author}
%\address{Department of Mathematics, University of South Carolina,
%Columbia, SC 29208}
%\email{second@math.sc.edu}
%\urladdr{www.math.sc.edu/$\sim$second}

%%
%% If there are three of more authors they are added in the obvious
%% way. 
%%

%%%
%%% The following is for the abstract.  The abstract is optional and
%%% if not used just delete, or comment out, the following.
%%%

\begin{abstract}
Great stuff.
\end{abstract}

%%
%%  LaTeX will not make the title for the paper unless told to do so.
%%  This is done by uncommenting the following.
%%

% \maketitle

%%
%% LaTeX can automatically make a table of contents.  This is done by
%% uncommenting the following:
%%

%\tableofcontents

%%
%%  To enter text is easy.  Just type it.  A blank line starts a new
%%  paragraph. 
%%


Call me Ishmael. Some years ago --- never mind how long precisely ---
having little or no money in my purse, and nothing particular to
interest me on shore, I thought I would sail about a little and see
the watery part of the world. It is a way I have of driving off the
spleen, and regulating the circulation.  Whenever I find myself
growing grim about the mouth; whenever it is a damp, drizzly November
in my soul; whenever I find myself involuntarily pausing before coffin
warehouses, and bringing up the rear of every funeral I meet; and
especially whenever my hypos get such an upper hand of me, that it
requires a strong moral principle to prevent me from deliberately
stepping into the street, and methodically knocking people's hats off
--- then, I account it high time to get to sea as soon as I can. This
is my substitute for pistol and ball. With a philosophical flourish
Cato throws himself upon his sword; I quietly take to the ship. There
is nothing surprising in this. If they but knew it, almost all men in
their degree, some time or other, cherish very nearly the same
feelings towards the ocean with me.


There now is your insular city of the Manhattoes, belted round by
wharves as Indian isles by coral reefs - commerce surrounds it with
her surf. Right and left, the streets take you waterward. Its extreme
down-town is the battery, where that noble mole is washed by waves,
and cooled by breezes, which a few hours previous were out of sight of
land. Look at the crowds of water-gazers there.


%%
%%  To put mathematics in a line it is put between dollor signs.  That
%%  is $(x+y)^2=x^2+2xy+y^2$
%%

Anyone caught using formulas such as $\sqrt{x+y}=\sqrt{x}+\sqrt{y}$ 
or $\frac{1}{x+y}=\frac{1}{x}+\frac{1}{y}$ will fail.

%%
%%% Displayed mathematics is put between double dollar signs.  
%%

The binomial theorem is
$$
(x+y)^n=\sum_{k=0}^n\binom{n}{k}x^ky^{n-k}.
$$
A favorite sum of most mathematicians is
$$
\sum_{n=1}^\infty \frac{1}{n^2}=\frac{\pi^2}{6}.
$$
Likewise a popular integral is
$$
\int_{-\infty}^\infty e^{-x^2}\,dx=\sqrt{\pi}
$$


%%
%% A Theorem is stated by
%%

\begin{thm} The square of any real number is non-negative.
\end{thm}

%%
%% Its proof is set off by
%% 

\begin{proof}
Any real number $x$ satisfies $x>0$, $x=0$, or $x<0$.
If $x=0$, then $x^2=0\ge 0$.  If $x>0$ then as a positive time a
positive is positive we have $x^2=xx>0$.  If $x<0$ then $-x>0$ and so
by what we have just done $x^2=(-x)^2>0$.  So in all cases $x^2\ge0$.
\end{proof}


%%
%% A new section is started as follows:
%%


%%%%%%%%%%%%%%%%%%%%%%%%%%%%%%%%%%%%%%%%%%%%%%%%%%%%%%%%%%%%%%%%%%%%%%
\section{Introduction}
%%%%%%%%%%%%%%%%%%%%%%%%%%%%%%%%%%%%%%%%%%%%%%%%%%%%%%%%%%%%%%%%%%%%%%

This is a new section.

%%
%% A subsection is started by
%%

\subsection{Things that need to be done.}
Prove theorems.




\end{document}






